
% Sample file for AES paper
\documentclass{aes2e}

\usepackage{amsmath}
\usepackage{fourier}
\DeclareMathAlphabet{\mathcal}{OMS}{cmsy}{m}{n}
\SetMathAlphabet{\mathcal}{bold}{OMS}{cmsy}{b}{n}

% Metadata Information
\jyear{2010}
\jmonth{October}
\jvol{1}
\jnum{1}


\begin{document}

% Page heads
\markboth{A1LASTNAM AND A2LASTNAME}{SPECTRAL DELAY FILTERS}


% Title portion
\title{Parallel Triangle Counting Algorithms\thanks{To whom correspondence should be addressed Tel: +1-240-381-2383; Fax: +1-202-508-3799; e-mail: info@schtm.org}}

%Author Info.
\authorgroup{
\author{Michele Carloni}
AND \author{Riccardo Revalor}
\email{(abc@abc.com)\quad\quad\quad\quad\quad\quad\quad\quad\quad\quad\quad\quad (s339423@studenti.polito.it)}
\affil{POlitecnico di Torino, Computer Engineering}
}

%Abstract
\abstract{%
This paper discusses the implementation of spectral delay using
filters comprising a cascade of many low-order allpass filters and an
equalizing filter. The spectral delay filters have chirp-like impulse
responses causing a large, frequency-dependent delay that is useful in
audio effects processing. An equalizing filter design and a multirate
technique, which stretches the allpass filters, impulse
response, are introduced.}


\maketitle

%Head 1
\section{Mathematical definition of the problem}
Suppose that we have at our disposal a given graph $G = (V, E)$, where $G$ is the graph itself, $V$ denotes the set of all vertices (also called nodes), and $E$ represents the set of edges (also called links or arcs). An edge connects two nodes between them. The primary goal is to detect and count the number of triangles that can be formed inside the graph. More specifically, a \textbf{triangle} in graph $G = (V, E)$ is a set of three distinct vertices $\{u, v, w\} \subseteq V$ such that the edges $(u,v)$, $(v,w)$, and $(w,u)$ all exist in the set of edges $E$. This problem can be solved essentially using two distinct main approaches:
\subsection{Graph Traversal/Iterative Methods}
This category encompasses algorithms like the \textbf{Node Iterator} or the \textbf{Edge Iterator} that directly explore the graph, and at each iteration they try to look for the specific pattern that defines a triangle (three mutually connected vertices, as said before). They operate by iterating through parts of the graph (by scanning the list of either its nodes or edges) and checking for connections.
\subsection{Linear Algebra/Matrix Methods}
In this category, we mainly recall the \textbf{Matrix Multiplication} approach, which only works if the graph is represented by an adjacency matrix, which leverages matrix operations, specifically cubing (i.e. matrix multiplication) and trace computation, to get the exact number of triangles which can be created. Given the graph adjacency matrix $A$, this number is computed as $$\frac{1}{6} \text{ Trace}(A^3)$$ In graph terms, this trace represents the total count of all closed walks of length 3 in the graph, summed across all possible starting/ending vertices. 

\section{Node Iterator Algorithm, Forward Algorithm}
The simplest and most naive version of the Node Iterator Algorithm iterates over all nodes in the graph and, for each pair of neighbors of each node, checks whether they are connected by an edge. For each vertex, its \textbf{degree} $d(v)$ denotes its neighbors. If we have $V$ vertices, for each of them, we test unique pairs of neighbors that can be formed from $d(v)$ items. So, the overall complexity can be expressed as
\begin{align*}
\text{Complexity} &= \sum_{v \in V} \binom{d(v)}{2} \\
&= \sum_{v \in V} \frac{d(v)!}{2!(d(v)-2)!} \\
&= \sum_{v \in V} \frac{d(v) \cdot (d(v)-1) \cdot (d(v)-2)!}{2 \cdot 1 \cdot (d(v)-2)!} \\
&= \sum_{v \in V} \frac{d(v) \cdot (d(v)-1)}{2}
\end{align*}
In this project, we resorted to the so called \textbf{Forward Algorithm}, a variant which aims at improving complexity by iterating over vertices in a specific order, and efficiently queries their neighbors to identify potential triangle closures. We first implemented the \textbf{sequential} version of the algorithm, using \textbf{C++}, in which we can identify three main data structures used:
\begin{itemize}
    \item \textbf{Graph Representation:} How the graph's vertices and edges are stored, enabling efficient retrieval of a node's neighbors. We used and analyzed two primary representations: \textbf{adjacency matrices} (implemented as \texttt{std::vector<std::vector<int>>}) and \textbf{adjacency lists} (implemented as \texttt{std::map<int, std::vector<int>>} to handle potentially sparse or non-contiguous node IDs).
    \item \textbf{Auxiliary Sets (\textit{A[t]}):} Data structures used to store a specific subset of neighbors for each node (implemented as \texttt{std::vector<std::set<int>>}, where the outer vector is indexed by the node ID). These sets are then intersected (as defined in set theory) to identify the third vertex of a triangle.
    \item \textbf{Ordering/Ranks:} Structures (a sorted list of vertices and a map for ranks) to maintain and query the chosen ordering of vertices. Our implementation orders vertices by their degree (from highest to lowest) to optimize performance.
\end{itemize}
In the file \texttt{seq\_node\_it\_v1.cpp} there's the implementation using the adjacency matrix. An entry \texttt{adjacencyMatrix[i][j]} is 1 if there is an edge between vertex \texttt{i} and vertex \texttt{j}, and 0 otherwise. So, with N nodes, the space complexity to store the matrix is $\mathcal{O}\left(N^2\right)$. Neighbor retrieval is linear, as it's mandatory to scan the full row of the node inside the matrix. The auxiliary set is defined as \texttt{std::vector<std::set<int>> A}. When processing an edge \texttt{(s, t)}, the algorithm needs to find common elements between \texttt{A[s]} and \texttt{A[t]}. As \texttt{std::set} stores elements in sorted order, \texttt{std::set\_intersection} can be efficiently applied, resulting in a complexity of $\mathcal{O}\left(K_0 + K_1\right)$, being $K_0$ and $K_1$ the sets sizes. For the orderings we use \texttt{std::map<int, int> ranks} and \texttt{std::vector<int> orderedList}.Checking rank condition takes $\mathcal{O}\left(\log N\right)$ for map lookups, and wwe iterate through the whole \texttt{orderedList} so it takes $\mathcal{O}\left( N\right)$. \\
In the file \texttt{seq\_node\_it\_v2.cpp} we used the adjacency list, coded as a \texttt{std::map<int, std::vector<int>>} to better handle potentially sparse or non-contiguous node IDs. This version is \textbf{more efficient}: to find the neighbors of a vertex u, the algorithm performs a lookup in the map ($\mathcal{O}\left(\log N\right)$) and then copies the associated vector of neighbors ($\mathcal{O}\left( d(u)\right)$). This is significantly more efficient for sparse graphs than iterating an entire row of an adjacency matrix. \\
So, to resume the comparison between the two algorithms:
\begin{table}[htbp]
    \centering
    \caption{Summary of Sequential Forward Algorithm Implementation Complexities}
    \label{tab:forward_complexity_summary}
    \begin{tabular}{|l|l|}
        \hline
        \textbf{Graph Representation} & \textbf{Overall Time Complexity} \\
        \hline
        Adjacency Matrix & $\mathcal{O}(N^3 \log N)$ \\
        Adjacency List & $\mathcal{O}(M \cdot d_{max})$ \\
        \hline
    \end{tabular}
    \vspace{0.5em} % Adds a little space below the table
    \small
    Where $N$ is the number of vertices, $M$ is the number of edges, and $d_{max}$ is the maximum degree in the graph.
\end{table}
%Head 2
\subsection{Chirp-Like Impulse Responses and Group Delay}
Filtering an audio signal with an allpass filter does not usually have a major effect on the signal's timbre. The allpass filter does not change the frequency content of the signal, but only introduces a phase shift or delay. Audibility of the phase distortion caused by an allpass filter in a sound reproduction system has been a topic of many studies, see, e.g., \cite{DEK1}, \cite{DEK2}. In this paper, we investigate audio effects processing using high-order allpass filters that consist of many cascaded low-order allpass filters. These filters have long chirp-like impulse responses. When audio and music signals are processed with such a filter, remarkable changes are obtained that are similar to the spectral delay effect  \cite{DEK3}, \cite{DEK4}.

%Head 3
\subsubsection{Chirp-Like Impulse Responses and Group Delay}
Filtering an audio signal with an allpass filter does not usually have a major effect on the signal's timbre. The allpass filter does not change the frequency content of the signal, but only introduces a phase shift or delay. Audibility of the phase distortion caused by an allpass filter in a sound reproduction system has been a topic of many studies, see, e.g., \cite{DEK1}, \cite{DEK2}. In this paper, we investigate audio effects processing using high-order allpass filters that consist of many cascaded low-order allpass filters.  When audio and music signals are processed with such a filter, remarkable changes are obtained that are similar to the spectral delay effect  \cite{DEK3}, \cite{DEK4}.
%Equation
\begin{equation}
A(z) = \frac{{a_1  + z^{ - 1} }}{{1 + a_1 z^{ - 1} }},
\end{equation}



Filtering an audio signal with an allpass filter does not usually have a major effect on the signal's timbre. The allpass filter does not change the frequency content of the \nobreak signal, but only introduces a phase shift or delay.\footnote{This point is emphasized by Loewer, see esp. p. (610).} Audibility of the phase distortion caused by an allpass filter in a sound reproduction system has been a topic of many studies, see, e.g., \cite{DEK1}, \cite{DEK2}. In this paper, we investigate audio effects processing using high-order allpass filters that consist of many cascaded low-order allpass filters. These filters have long chirp-like impulse responses. When audio and music signals are processed with such a filter, remarkable changes are obtained that are similar to the spectral delay effect  \cite{DEK3}, \cite{DEK4}.
\begin{equation}
\tau _{\textrm{g,max}}  = \left\{ \begin{array}{l}
 \tau _\textrm{g} (0) = \frac{{1 - a_1 }}{{1 + a_1 }},\textrm{when }a_1  \le 0 \\[4pt]
 \tau _\textrm{g} (\pi ) = \frac{{1 + a_1 }}{{1 - a_1 }},\textrm{when }a_1  > 0. \\
 \end{array} \right.\end{equation}

Filtering an audio signal with an allpass filter does not usually have a major effect on the signal's timbre. The allpass filter does not change the frequency content of the signal, but only introduces a phase shift or delay. Audibility of the phase distortion caused by an allpass filter in a sound reproduction system has been a topic of many studies, see, e.g., \cite{DEK1}, \cite{DEK2}. In this paper, we investigate audio effects processing using high-order allpass filters that consist of many cascaded low-order allpass filters. These filters have long chirp-like impulse responses. When audio and music signals are processed with such a filter, remarkable changes are obtained that are similar to the spectral delay effect  \cite{DEK3}, \cite{DEK4}.
%Paragraph listing
\begin{paralist}
\item{}Filtering an audio signal with an allpass filter does not usually have a major effect on the signal's timbre. The allpass filter does not change the frequency content of the signal, but only introduces a phase shift or delay. 
\item{}Audibility of the phase distortion caused by an allpass filter in a sound reproduction system has been a topic of many studies, see, e.g., \cite{DEK1}, \cite{DEK2}. 
\item{}In this paper, we investigate audio effects processing using high-order allpass filters that consist of many cascaded low-order allpass filters. These filters have long chirp-like impulse responses. 
\item{}When audio and music signals are processed with such a filter, remarkable changes are obtained that are similar to the spectral delay effect  \cite{DEK3}, \cite{DEK4}.
\end{paralist}

Filtering an audio signal with an allpass filter does not usually have a major effect on the signal's timbre. The allpass filter does not change the frequency content of the signal, but only introduces a phase shift or delay. Audibility of the phase distortion caused by an allpass filter in a sound reproduction system has been a topic of many studies, see, e.g., \cite{DEK1}, \cite{DEK2}. In this paper, we investigate audio effects processing using high-order allpass filters that consist of many cascaded low-order allpass filters. These filters have long chirp-like impulse responses. When audio and music signals are processed with such a filter, remarkable changes are obtained that are similar to the spectral delay effect  \cite{DEK3}, \cite{DEK4}.
\begin{arabiclist}
\item{}Green--function determined experimentally and published.
\item{}Black--function determined using similarity searches and published.
\item{}Red--function determined using similarity searches and determined in this study.
\item{}Blue--O-antigen structure unknown. Function determined using similarity searches and proposed in this study.
\end{arabiclist}

%Table
\begin{table}
\tabcolsep8.1pt
\tbl{Active sites and allosteric sites of the GNE MNK enzyme}{%
\begin{tabular}{@{}lccc@{}}\toprule
Excerpt No.& Genre & Spatial Mode & Corrlation\\\colrule
 1 & Pop       & FB   & 94\%\\
 2 & Classical & FB   & 33\%\\
 3 & Jazz      & FF   & 76\%\\
 4 & Arabian   & FF   & 41\%\\
 5 & GNE       & H220 & 45\%\\
 6 & GNE       & H45  & 93\%\\
 7 & MNK       & G416 & 74\%\\
 8 & MNK       & D413 & 72\%\\
 9 & MNK       & R420 & 94\%\\
10 & MNK       & N516 & 91\%\\\botrule
\end{tabular}}
\begin{tabnote}
Note. This table does not include sentence enhancement statutes.  This table does not include sentence enhancement statutes.
\end{tabnote}
\end{table}

%Figure
\begin{figure}
\centering
\includegraphics{aes2e-mouse.eps}
\caption{The spectral delay filter consists of \textit{M} allpass filters and an equalization filter.}
\end{figure}


\begin{figure*}
\centering
\includegraphics[width=23pc]{aes2e-mouse.eps}
\caption{This paper is organized as follows. In Section 1, we discuss the group delay of a cascade of first-order allpass filters and its relation to the chirp-like impulse response of the spectral delay filter. Furthermore, a multirate method to stretch the impulse response of the spectral delay filter is proposed. Section 2 discusses the amplitude envelope of the impulse response and suggests a design method for the equalizing filter. Section 3 presents application examples using the spectral delay filter. Section 4 concludes this paper.}
\end{figure*}
Filtering an audio signal with an allpass filter does not usually have a major effect on the signal's timbre. The allpass filter does not change the frequency content of the signal, but only introduces a phase shift or delay.
\begin{extract}
Filtering an audio signal with an allpass filter does not usually have
a major effect on the signal's timbre. The allpass filter does not
change the frequency content of the signal, but only introduces a
phase shift or delay. Audibility of the phase distortion caused by an
allpass filter in a sound reproduction system has been a topic of many
studies, see, e.g., \cite{DEK1}, \cite{DEK2}. In this paper, we
investigate audio effects \nobreak processing using high-order allpass filters that consist of many cascaded low-order allpass filters. These filters have long chirp-like impulse responses. When audio and music signals are processed with such a filter, remarkable changes are obtained that are similar to the spectral delay effect  \cite{DEK3}, \cite{DEK4}.
\end{extract}
Filtering an audio signal with an allpass filter does not usually have a major effect on the signal's timbre. The allpass filter does not change the frequency content of the signal, but only introduces a phase shift or delay. 
\[
\tau _\textrm{g} (\omega ) =  - \frac{{d\phi (\omega )}}{{d\omega }}.
\]
Audibility of the phase distortion caused by an allpass filter in a sound reproduction system has been a topic of many studies, see, e.g., \cite{DEK1}, \cite{DEK2}. In this paper, we investigate audio effects processing using high-order allpass filters that consist of many cascaded low-order allpass filters. These filters have long chirp-like impulse responses. When audio and music signals are processed with such a filter, remarkable changes are obtained that are similar to the spectral delay effect.
\begin{alphalist}
\item{}Green--function determined experimentally and published.
\item{}Black--function determined using similarity searches and published.
\item{}Red--function determined using similarity searches and determined in this study.
\item{}Blue--O-antigen structure unknown. Function determined using similarity searches and proposed in this study.
\end{alphalist}
Filtering an audio signal with an allpass filter does not usually have a major effect on the signal's timbre. The allpass filter does not change the frequency content of the signal, but only introduces a phase shift or delay. Audibility of the phase distortion caused by an allpass filter in a sound reproduction system has been a topic of many studies, see, e.g., \cite{DEK1}, \cite{DEK2}. 
%Enunciations
\begin{example}
In this paper, we investigate audio effects processing using high-order allpass filters that consist of many cascaded low-order allpass filters. These filters have long chirp-like impulse responses. 
\end{example}
Filtering an audio signal with an allpass filter does not usually have a major effect on the signal's timbre. The allpass filter does not change the frequency content of the signal, but only introduces a phase shift or delay. Audibility of the phase distortion caused by an allpass filter in a sound reproduction system has been a topic of many studies.
\begin{bulletlist}
\item{}Green--function determined experimentally and published.
\item{}Black--function determined using similarity searches and published.
\item{}Red--function determined using similarity searches and determined in this study.
\item{}Blue--O-antigen structure unknown. Function determined using similarity searches and proposed in this study.
\end{bulletlist}
Filtering an audio signal with an allpass filter does not usually have a major effect on the signal's timbre. The allpass filter does not change the frequency content of the signal, but only introduces a phase shift or delay. Audibility of the phase distortion caused by an allpass filter in a sound reproduction system has been a topic of many studies, see, e.g., \cite{DEK1}, \cite{DEK2}. In this paper, we investigate audio effects processing using high-order allpass filters that consist of many cascaded low-order allpass filters. These filters have long chirp-like impulse responses. When audio and music signals are processed with such a filter, remarkable changes are obtained that are similar to the spectral delay effect  \cite{DEK3}, \cite{DEK4}.
\begin{unnumlist}
\item{}Green--function determined experimentally and published.
\item{}Black--function determined using similarity searches and published.
\item{}Red--function determined using similarity searches and determined in this study.
\item{}Blue--O-antigen structure unknown. Function determined using similarity searches and proposed in this study.
\end{unnumlist}
Filtering an audio signal with an allpass filter does not usually have a major effect on the signal's timbre. The allpass filter does not change the frequency content of the signal, but only introduces a phase shift or delay. Audibility of the phase distortion caused by an allpass filter in a sound reproduction system has been a topic of many studies, see, e.g., \cite{DEK1}, \cite{DEK2}. In this paper, we investigate audio effects processing using high-order allpass filters that consist of many cascaded low-order allpass filters. These filters have long chirp-like impulse responses. When audio and music signals are processed with such a filter, remarkable changes are obtained that are similar to the spectral delay effect  \cite{DEK3}, \cite{DEK4}.

\section{SUMMARY}
Filtering an audio signal with an allpass filter does not usually have a major effect on the signal's timbre. The allpass filter does not change the frequency content of the signal, but only introduces a phase shift or delay. Audibility of the phase distortion caused by an allpass filter in a sound reproduction system has been a topic of many studies, see, e.g., \cite{DEK1}, \cite{DEK2}. In this paper, we investigate audio effects processing using high-order allpass filters that consist of many cascaded low-order allpass filters. These filters have long chirp-like impulse responses. When audio and music signals are processed with such a filter, remarkable changes are obtained that are similar to the spectral delay effect  \cite{DEK3}, \cite{DEK4}.

\section{CONCLUSION}
Filtering an audio signal with an allpass filter does not usually have a major effect on the signal's timbre. The allpass filter does not change the frequency content of the signal, but only introduces a phase shift or delay. Audibility of the phase distortion caused by an allpass filter in a sound reproduction system has been a topic of many studies, see, e.g., \cite{DEK1}, \cite{DEK2}. In this paper, we investigate audio effects processing using high-order allpass filters that consist of many cascaded low-order allpass filters. These filters have long chirp-like impulse responses. When audio and music signals are processed with such a filter, remarkable changes are obtained that are similar to the spectral delay effect  \cite{DEK3}, \cite{DEK4}. Note that articles might have a digital object identifier~\cite{DEK5}.

\section{ACKNOWLEDGMENT}
This research was conducted in fall 2008 when Vesa V\"alim\"aki was a visiting scholar at CCRMA, Stanford University. His visit was financed by the Academy of Finland (project no. 126310). The authors would like to Dr. Henri Penttinen for his comments and for the snare drum sample used in this work.

\bibliography{aes2e.bib}
\bibliographystyle{aes2e.bst}

% NOTE:
% - in case you are not using bibTex you have to manually edit the bibliograpy as below.
% - if submitting a bibTex file is not allowed you can copy the content from the aes2e.bbl file  
%\begin{thebibliography}{99}
%
%\newcommand{\enquote}[1]{``#1''}
%\providecommand{\url}[1]{\texttt{#1}}
%\providecommand{\urlprefix}{URL }
%\expandafter\ifx\csname urlstyle\endcsname\relax
%  \providecommand{\doi}[1]{[Online]. Available: \discretionary{}{}{}#1}\else
%  \providecommand{\doi}{doi:\discretionary{}{}{}\begingroup
%  \urlstyle{rm}\Url}\fi
%
%\bibitem{DEK1}
%D.~Preis, \enquote{Phase Distortion and Phase Equalization in Audio Signal
%  Processing---A Tutorial Review,} \emph{J. Audio Eng. Soc.}, vol.~30, no.~11,
%  pp. 774--779 (1982 Nov.).
%
%\bibitem{DEK2}
%J.~S. Abel, D.~P. Berners, \enquote{MUS424/EE367D: Signal Processing Techniques
%  for Digital Audio Effects,}  (2005), unpublished Course Notes, CCRMA,
%  Stanford University, Stanford, CA.
%
%\bibitem{DEK3}
%C.~Roads, \enquote{Musical Sound Transformation by Convolution,} presented at
%  the \emph{Int. Computer Music Conf.}, pp. 102--109 (1993).
%
%\bibitem{DEK4}
%C.~Roads, \emph{The Computer Music Tutorial} (MIT Press, Cambridge, MA), 1st
%  ed. (1996).
%
%\bibitem{DEK5}
%H.~Morgenstern, B.~Rafaely, \enquote{Spatial Reverberation and Dereverberation
%  Using an Acoustic Multiple-Input Multiple-Output System,} \emph{J. Audio Eng.
%  Soc}, vol.~65, no. 1/2, pp. 42--55 (2017 Jan.Feb.),
%  \doi{https://doi.org/10.17743/jaes.2016.0063}.
%  
%\end{thebibliography}

%Appendix
\appendix
\section*{APPENDIX}
Filtering an audio signal with an allpass filter does not usually have a major effect on the signal's timbre. The allpass filter does not change the frequency content of the signal, but only introduces a phase shift or delay. Audibility of the phase distortion caused by an allpass filter in a sound reproduction system has been a topic of many studies, see, e.g., \cite{DEK1}, \cite{DEK2}.
\begin{equation}
\phi (\omega ) =  - \omega  + 2\arctan \left( {\frac{{a_1 \sin \omega }}{{1 + a_1 \cos \omega }}} \right)
\end{equation}

In this paper, we investigate audio effects processing using high-order allpass filters that consist of many cascaded low-order allpass filters. These filters have long chirp-like impulse responses. When audio and music signals are processed with such a filter, remarkable changes are obtained that are similar to the spectral delay effect  \cite{DEK3}, \cite{DEK4}.


\begin{nomenclature}[PAMPs]
\subsection*{NOMENCLATURE}
\nomentry{a$_c$}{condensation coefficient condensation coefficient condensation coefficient}


\nomentry{TLR}{Toll-like receptor}

\nomentry{PAMPs}{pathogen-associated molecular patterns condensation coefficient condensation}
\end{nomenclature}

%Biography
 \biography{A1firstname A1lastname}{a.eps}{A1firstname A1lastname is professor of audio signal processing at Helsinki University of Technology (TKK), Espoo, Finland. He received his Master of Science in Technology, Licentiate of Science in Technology, and Doctor of Science in Technology degrees in electrical engineering from TKK in 1992, 1994, and 1995, respectively. His doctoral dissertation dealt with fractional delay filters and physical modeling of musical wind instruments. Since 1990, he has worked mostly at TKK with the exception of a few periods. In 1996 he spent six months as a postdoctoral research fellow at the University of Westminster, London, UK. In 2001-2002 he was professor of signal processing at the Pori School of Technology and Economics, Tampere University of Technology, Pori, Finland. During the academic year 2008-2009 he has been on sabbatical and has spent several months as a visiting scholar at the Center for Computer Research in Music and Acoustics (CCRMA), Stanford University, Stanford, CA. His research interests include musical signal processing, digital filter design, and acoustics of musical instruments. Prof. V\"alim\"aki is a senior member of the IEEE Signal Processing Society and is a member of the AES, the Acoustical Society of Finland, and the Finnish Musicological Society. He was the chairman of the 11th International Conference on Digital Audio Effects (DAFx-08), which was held in Espoo, Finland, in 2008.}
 \biography{A2firstname A2lastname}{b.eps}{A2firstname A2lastname is a consulting professor at the Center for Computer Research in Music and Acoustics (CCRMA) in the Music Department at Stanford University where his research interests include audio and music applications of signal and array processing, parameter estimation, and acoustics. From 1999 to 2007, Abel was a co-founder and chief technology officer of the Grammy Award-winning Universal Audio, Inc. He was a researcher at NASA/Ames Research Center, exploring topics in room acoustics and spatial hearing on a grant through the San Jose State University Foundation. Abel was also chief scientist of Crystal River Engineering, Inc., where he developed their positional audio technology, and a lecturer in the Department of Electrical Engineering at Yale University. As an industry consultant, Abel has worked with Apple, FDNY, LSI Logic, NRL, SAIC and Sennheiser, on projects in professional audio, GPS, medical imaging, passive sonar and fire department resource allocation. He holds Ph.D. and M.S. degrees from Stanford University, and an S.B. from MIT, all in electrical engineering. Abel is a Fellow of the Audio Engineering Society.}
\end{document}